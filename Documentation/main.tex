% das Papierformat zuerst
\documentclass[a4paper, 11pt, oneside]{article}

% deutsche Sonderzeichen benutzen
\usepackage[ngerman]{babel}


\usepackage[ansinew]{inputenc} % german language signes
\usepackage{amsmath,amssymb,amsfonts,amsthm}    % Typical maths resource packages

\usepackage{graphicx}

\usepackage{hyperref}                           % For creating hyperlinks in cross references

\usepackage{tocbibind}
\usepackage{fancybox,amssymb,color}
\usepackage{oubraces}

\usepackage{ltxtable, tabularx, longtable} % important for tables

\usepackage{glossaries}
\makeglossaries


%\usepackage[authoryear]{natbib}                 % literature reference style
%\usepackage[bf]{caption2}

%\newpsobject{showgrid}{psgrid}{subgriddiv=1,griddots=10,gridlabels=0pt}

\pagestyle{headings}

% hat was mit Abstaenden zu tun
\frenchspacing

% hier beginnt das Dokument

%Begriffdefinition
%\newglossaryentry{apf}{name=Apfel, description={Ost aus der Gruppe der Kernobstgew�chse}} 
\loadglsentries{99appendix/glossary.tex}

\begin{document}

%
% --- Title Page ---
%
\thispagestyle{empty}
\begin{center}

\begin{verbatim}

\end{verbatim}

\begin{center}
\Large{UNIGIS Professional - Projektarbeit}
\end{center}
\begin{verbatim}









\end{verbatim}
\begin{center}
\textbf{\LARGE{TBD}}
\end{center}
\begin{center}
\textbf{Bericht zur Umsetzung}
\end{center}
\begin{verbatim}


\end{verbatim}
\begin{verbatim}












\end{verbatim}

\begin{flushleft}
\begin{tabular}{lll}
& & \\
& & \\
& & \\
& & \\
\textbf{Eingereicht von:} & & Martin Stypinski \\
& & \\
& & \\
& & \\
\textbf{Abgegeben am:} & & tbd, 99.99.9999\\
& & \\
& & \\
\textbf{Betreuer:} & & tbd
\end{tabular}
\end{flushleft}
\end{center}


%
% --- Eigenst�dnigkeitserkl�rung ---
%
\newpage
\pagestyle{plain}
\pagenumbering{roman}   % define page number in roman style
\setcounter{page}{1}    % start page numbering
\thispagestyle{empty}
\addcontentsline{toc}{section}{Eigenst�ndigkeitserkl�rung}
\section*{Eigenst�ndigkeitserkl�rung}

TBD
\begin{verbatim}




\end{verbatim}
Binz, den 99. Mai 9999
\begin{verbatim}





\end{verbatim}
\begin{tabular*}{\textwidth}{m{2cm}m{7cm}m{7cm}m{2cm}}
&  & Martin Stypinski & \\
\end{tabular*}

%
% --- Abstract / Einleitung ---
%

\newpage
\addcontentsline{toc}{section}{Abstract}
\section*{Abstract}
Der Trend der heutigen Zeit ist eine starke Verschiebung vieler Desktop Anwendungen in das Internet. Doch wo sind die Grenzen dieses Trends? Welche Softwareaufgaben kann man heute in einem Webbrowser l�sen? Diese Frage wurde zu Beginn dieser Arbeit in den Vordergrund gestellt und so ist die Idee einer WebGIS L�sung aufgekommen um eine komplexe Aufgabe ganz ohne GIS Spezialwissen einem Anwender zug�nglich zu machen.
\\
In diesem Projekt wird eine WebGIS Anwendung konzepiert und Entwickelt um Standortbestimmungen und Standortvorschl�ge f�r ein bestehendes Filialen Netz zu erstellen. Der Anwender soll dabei m�glichst ohne spezifische GIS Kenntnisse in der Lage sein ein bestehendes Filialen Netz in die Anwendung zu laden und mit Hilfe von verschiedenen Operationen und Visualisierungshilfen schl�sse aus ihnen zu ziehen. Die Resultate welche mithilfe der Anwendung gezogen werden k�nnen, sollen helfen potentielle Standorte zu bestimmen. Im Vordergrund steht hierbei die \gls{Isochrone}n-Berechnung.
\\
Die Arbeit gliedert sich somit auf 2 grosse Teile. Einerseits soll die Problemstellung so anwenderfreundlich wie m�glich gel�st werden, andererseits ist die Wahl der Softwarearchitektur ein sehr wichtiger Bestandteil der Arbeit. Die Ausarbeitung der Softwarearchitektur sollte Aufschluss �ber einen geeigneten WebGIS \gls{Full Stack} geben.



%
% --- Inhtaltsverzeichnis ---
%
\newpage
\tableofcontents
\clearpage



%
% --- Inhalt der Arbeit ---
%
\newpage
\pagestyle{plain}
\setcounter{page}{1}    % start page numbering anew
\pagenumbering{arabic}  % page numbers in arabic style

\addcontentsline{toc}{section}{Architektur}
\section{Architektur}
\subsection{Einleitung}
Bei der Ausarbeitung der Architektur wurde versucht so stark wie mögliche auf OpenSource Komponenten zu setzten. Diese haben den Vorteil, dass sie kostengünstig zu betreiben und sehr transparent in der Entwicklung sind. Weiter standen Überlegungen wie eine modulare Lösung im Vordergrund.



\newpage

\section{Einleitung}
\gls{apf}

%
% --- Anhang ---
%

\appendix

% literature
\newpage
\addcontentsline{toc}{section}{References}
\bibliography{literature}

\newpage
\printglossary[title=Glossar]


\end{document}