\addcontentsline{toc}{section}{Abstract}
\section*{Abstract}
Der Trend der heutigen Zeit ist eine starke Verschiebung vieler Desktop Anwendungen in das Internet. Doch wo sind die Grenzen dieses Trends? Welche Softwareaufgaben kann man heute in einem Webbrowser l�sen? Diese Frage wurde zu Beginn dieser Arbeit in den Vordergrund gestellt und so ist die Idee einer WebGIS L�sung aufgekommen um eine komplexe Aufgabe ganz ohne GIS Spezialwissen einem Anwender zug�nglich zu machen.
\\
In diesem Projekt wird eine WebGIS Anwendung konzepiert und Entwickelt um Standortbestimmungen und Standortvorschl�ge f�r ein bestehendes Filialen Netz zu erstellen. Der Anwender soll dabei m�glichst ohne spezifische GIS Kenntnisse in der Lage sein ein bestehendes Filialen Netz in die Anwendung zu laden und mit Hilfe von verschiedenen Operationen und Visualisierungshilfen schl�sse aus ihnen zu ziehen. Die Resultate welche mithilfe der Anwendung gezogen werden k�nnen, sollen helfen potentielle Standorte zu bestimmen. Im Vordergrund steht hierbei die \gls{Isochrone}n-Berechnung.
\\
Die Arbeit gliedert sich somit auf 2 grosse Teile. Einerseits soll die Problemstellung so anwenderfreundlich wie m�glich gel�st werden, andererseits ist die Wahl der Softwarearchitektur ein sehr wichtiger Bestandteil der Arbeit. Die Ausarbeitung der Softwarearchitektur sollte Aufschluss �ber einen geeigneten WebGIS \gls{Full Stack} geben.

